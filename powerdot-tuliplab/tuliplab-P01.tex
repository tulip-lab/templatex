% 
% ---------------------------------------------------------------
% Copyright (C) 2012-2018 Gang Li
% ---------------------------------------------------------------
%
% This work is the default powerdot-tuliplab style test file in TULIP Lab and may be
% distributed and/or modified under the conditions of the LaTeX Project Public
% License, either version 1.3 of this license or (at your option) any later
% version. The latest version of this license is in
% http://www.latex-project.org/lppl.txt and version 1.3 or later is part of all
% distributions of LaTeX version 2003/12/01 or later.
%
% This work has the LPPL maintenance status "maintained".
%
% This Current Maintainer of this work is Gang Li.
%
%

% Document Class
\documentclass[
 size=12pt,
 paper=smartboard,  %a4paper, smartboard, screen
 mode=present, 		%present, handout, print
 display=slides, 	% slidesnotes, notes, slides
 style=tuliplab,  	% TULIP Lab style
 pauseslide,
 fleqn,leqno]{powerdot}

\usepackage{amssymb}
\usepackage{amsmath} 
\usepackage{rotating}
\usepackage{graphicx}
\usepackage{boxedminipage}
\usepackage{media9}
\usepackage{rotate}
\usepackage{calc}
\usepackage[absolute]{textpos}
\usepackage{psfrag,overpic}
\usepackage{fouriernc}
\usepackage{pstricks,pst-node,pst-text,pst-3d,pst-grad}
\usepackage{moreverb,epsfig,color,subfigure}
\usepackage{color}
\usepackage{pstricks}
\usepackage{pstricks-add}
\usepackage{pst-text}
\usepackage{pst-node, pst-tree}
\usepackage{booktabs}
\usepackage{etex}
\usepackage{breqn}
\usepackage{multirow}
\usepackage{gitinfo2}
% \usepackage{pst-rel-points}

\usepackage{listings}
\lstset{frameround=fttt, 
frame=trBL, 
stringstyle=\ttfamily,
backgroundcolor=\color{yellow!20},
basicstyle=\footnotesize\ttfamily}
\lstnewenvironment{code}{
\lstset{frame=single,escapeinside=`',
backgroundcolor=\color{yellow!20},
basicstyle=\footnotesize\ttfamily}
}{}


\usepackage{hyperref}
\hypersetup{ % TODO: PDF meta Data
  pdftitle={Presentation Title},
  pdfauthor={Gang Li},
  pdfpagemode={FullScreen},
  pdfborder={0 0 0}
}


% \usepackage{auto-pst-pdf}
% package to show source code

\definecolor{LightGray}{rgb}{0.9,0.9,0.9}
\newlength{\pixel}\setlength\pixel{0.000714285714\slidewidth}
\setlength{\TPHorizModule}{\slidewidth}
\setlength{\TPVertModule}{\slideheight}
\newcommand\highlight[1]{\fbox{#1}}
\newcommand\icite[1]{{\footnotesize [#1]}}

\newcommand\twotonebox[2]{\fcolorbox{pdcolor2}{pdcolor2}
{#1\vphantom{#2}}\fcolorbox{pdcolor2}{white}{#2\vphantom{#1}}}
\newcommand\twotoneboxo[2]{\fcolorbox{pdcolor2}{pdcolor2}
{#1}\fcolorbox{pdcolor2}{white}{#2}}
\newcommand\vpspace[1]{\vphantom{\vspace{#1}}}
\newcommand\hpspace[1]{\hphantom{\hspace{#1}}}
\newcommand\COMMENT[1]{}

\newcommand\placepos[3]{\hbox to\z@{\kern#1
        \raisebox{-#2}[\z@][\z@]{#3}\hss}\ignorespaces} 


%%%%%%%%%%%%%%%%%%%%%%%%%%%%%%%%%%%%%%%%%%%%%%%%%%%%%%%%%%%%%%%%%%%%%%%%
% title
% TODO: Customize to your Own Title, Name, Address
%
\title{Presentation Title}
\author{
Gang Li
\\
Deakin University 
% \href{mailto:gangli@acm.org}{gangli@acm.org}
% \and % more authors
}
\date{\gitCommitterDate}


% Customize the setting of slides
\pdsetup{
% TODO: Customize the left footer, and right footer
rf=\href{http://www.tulip.org.au}{
Last Changed by: \textsc{\gitCommitterName}\ \gitVtagn-\gitAbbrevHash\ (\gitAuthorDate)
},
cf={Presentation Title},
}


% Starts the document
\begin{document}

\maketitle 

%%==========================================================================================
%%
\begin{slide}[toc=,bm=]{Table of Content}
 \tableofcontents[type=1,content=futuresections]
%  \tableofcontents[content=sections]
\end{slide}
%%
%%==========================================================================================

\section{Background}

\begin{slide}{Pause longer}
\begin{itemize}
\item A \pause
\item B \pause[2]
\item C
\end{itemize}
\end{slide}


\section{Aggregation Functions}


%%==========================================================================================
%%
\begin{slide}{Multi-Criteria Decision Making}
\twotonebox{Problem}{\parbox{.8\textwidth}
{A customer wants to select a product
that is \emph{inexpensive}, \emph{good quality},
and with \emph{good service}.
Choices have been narrowed down to three products
with following utility values:
}}
\smallskip
\begin{tabular}{ c| c c c | c }
\toprule
\centering
\texttt{Product}  & \texttt{Price} & \texttt{Quality} & \texttt{Service}  & Overall Rating \\
\midrule
\onslide*{2}{\textcolor{red}{$P_1$}} \onslide*{-1,3-}{{$P_1$}} 
&  \onslide*{2}{\textcolor{red}{$0.4$}} \onslide*{-1,3-}{{$0.4$}}   
&  \onslide*{2}{\textcolor{red}{$0.3$}} \onslide*{-1,3-}{{$0.3$}}   
&  \onslide*{2}{\textcolor{red}{$0.8$}} \onslide*{-1,3-}{{$0.8$}}   
& \onslide*{2}{\textcolor{red}{?}} \onslide*{-1,3-}{{?}}  \\
\onslide*{3}{\textcolor{red}{$P_2$}} \onslide*{-2,4-}{{$P_2$}} 
&  \onslide*{3}{\textcolor{red}{$0.1$}} \onslide*{-2,4-}{{$0.1$}} 
&  \onslide*{3}{\textcolor{red}{$0.6$}} \onslide*{-2,4-}{{$0.6$}} 
&  \onslide*{3}{\textcolor{red}{$0.5$}} \onslide*{-2,4-}{{$0.5$}} 
& \onslide*{3}{\textcolor{red}{?}}  \onslide*{-2,4-}{{?}}  \\
\onslide*{4}{\textcolor{red}{$P_3$}} \onslide*{-3,5-}{{$P_3$}} 
&  \onslide*{4}{\textcolor{red}{$0.6$}} \onslide*{-3,5-}{{$0.6$}} 
&  \onslide*{4}{\textcolor{red}{$0.4$}} \onslide*{-3,5-}{{$0.4$}} 
&  \onslide*{4}{\textcolor{red}{$0.3$}} \onslide*{-3,5-}{{$0.3$}} 
& \onslide*{4}{\textcolor{red}{?}} \onslide*{-3,5-}{{?}}  \\
\bottomrule
\end{tabular}
\bigskip
\begin{center}
 \onslide{5}{\twotonebox{\Huge ?}{\large \emph{Which is the \textcolor{red}{best} product}}}
\end{center}
\end{slide}
%%
%%==========================================================================================

%%==========================================================================================
%%
\begin{slide}[toc=,bm=]{Aggregation Functions --- WAM}
\twotonebox{Problem}{\parbox{.8\textwidth}
{A customer wants to select a product
that is \emph{inexpensive}, \emph{good quality},
and with \emph{good service}.
Choices have been narrowed down to three products
with following utility values:
}}
\smallskip
\begin{tabular}{ c| c c c | c }
\toprule
\centering
\texttt{Product}  & \texttt{Price} & \texttt{Quality} & \texttt{Service}  & Overall Rating \\
\midrule
\onslide*{3}{\textcolor{red}{$P_1$}} \onslide*{-2,4-}{{$P_1$}} 
&  \onslide*{3}{\textcolor{red}{$0.4$}} \onslide*{-2,4-}{{$0.4$}}   
&  \onslide*{3}{\textcolor{red}{$0.3$}} \onslide*{-2,4-}{{$0.3$}}   
&  \onslide*{3}{\textcolor{red}{$0.8$}} \onslide*{-2,4-}{{$0.8$}}   
& \onslide*{3}{\textcolor{red}{?}} \onslide*{4-}{{$0.475$}}  \\
\onslide*{4}{\textcolor{red}{$P_2$}} \onslide*{-3,5-}{{$P_2$}} 
&  \onslide*{4}{\textcolor{red}{$0.1$}} \onslide*{-3,5-}{{$0.1$}} 
&  \onslide*{4}{\textcolor{red}{$0.6$}} \onslide*{-3,5-}{{$0.6$}} 
&  \onslide*{4}{\textcolor{red}{$0.5$}} \onslide*{-3,5-}{{$0.5$}} 
& \onslide*{4}{\textcolor{red}{?}} \onslide*{5-}{{$0.325$}}  \\
\onslide*{5}{\textcolor{red}{$P_3$}} \onslide*{-4,6-}{{$P_3$}} 
&  \onslide*{5}{\textcolor{red}{$0.6$}} \onslide*{-4,6-}{{$0.6$}} 
&  \onslide*{5}{\textcolor{red}{$0.4$}} \onslide*{-4,6-}{{$0.4$}} 
&  \onslide*{5}{\textcolor{red}{$0.3$}} \onslide*{-4,6-}{{$0.3$}} 
& \onslide*{5}{\textcolor{red}{?}} \onslide*{6-}{{\rnode{t1}{$0.950$}}}  \\
\midrule
\onslide*{2-}{\texttt{Weights}}  
& \onslide*{2-}{$0.5$}	 
& \onslide*{2-}{$0.25$}	 
& \onslide*{2-}{$0.25$}	& \\
\bottomrule
\end{tabular}
\smallskip
\twocolumn[lcolwidth=.54\textwidth,rcolwidth=.34\textwidth]
{\scriptsize
\begin{itemize}[type=0]
	\item<2-> Weighted Arithmetic Mean 
% \tiny
\begin{description}[type=0]
 \item[$P_1$]<3-> $0.5(0.4)+0.25(0.3)+0.25(0.8) = 0.475$\\
 \item[$P_2$]<4-> $0.5(0.1)+0.25(0.6)+0.25(0.5) = 0.325$\\
 \item[$P_3$]<5-> $0.5(0.6)+0.25(0.4)+0.25(0.3) = 0.950$
\end{description}
\end{itemize}
}
{
\medskip %\vspace{.5cm}
\vspace{\stretch{1}}
\onslide*{7-}{\centering \twotonebox{?}{
\emph{the 
\onslide*{7}{\textcolor{red}{best}} 
\onslide*{8-}{\rnode[tc]{T1}{\textcolor{red}{best}} 
\psset{arrowscale=2,arrows=->}
\nccurve[linecolor=red,angleA=45,angleB=180]{T1}{t1}}
product}}}
}
\vspace{\stretch{1}}
\end{slide}
%%
%%==========================================================================================
%%==========================================================================================
%%  
\begin{slide}[toc={Choquet Integral},bm=]{Fuzzy Measure}
\twotonebox{\rotatebox{0}{Definition}}{\parbox{.8\textwidth}
{
A \emph{fuzzy measure} is a set of functions $v: 2^N \to [0,1]$
on all possible combinations of $n$ criteria,
which satisifies:
\begin{itemize}
 \item $v(A) \le v(B)$ whenever $A \subset B$;
 \item $v(\emptyset) = 0$ and $v(N) = 1$.
\end{itemize}  
}}
\medskip
\onslide*{2-}{ \begin{itemize}
\item  Since $A \subseteq N$ is a \emph{coalition} of criteria,
$v(A)$ represents the importance of this coalition.
\end{itemize}
}
\end{slide}
%%
%%==========================================================================================

%%==========================================================================================
%%  
\begin{slide}[toc=,bm=]{\emph{Choquet Integral} --- Summary}
\begin{itemize}[type=1]
\item<1-> The Choquet Integral generalises many important aggregation functions including 
the \emph{mean}, \emph{weighted arithmetic mean}, \emph{maximum}, 
\emph{minimum} and \emph{ordered weighted average}.
\begin{itemize}[type=1]
\item<2-> Unlike WAM which allocates a weight to each input,
it assigns a weight to all the subsets of inputs, 
hence $2^n$ parameters.
\end{itemize}
\item<3-> \emph{K-Additivity}:
$k$-additive fuzzy measures model no interaction of groups with more than $k$ criteria
\begin{itemize}[type=1]
 \item<4-> It allows an arbitrary reduction in complexity at the expense of modelling ability.
 \item<5-> In the case of $3$-addivitity,
we can model the interactions of pairs and triples,
but not of larger groups.
\end{itemize}
\end{itemize}
\end{slide}
%%
%%==========================================================================================


\section{Application and Analysis}

%%==========================================================================================
%%  
\begin{slide}{Data Collection}
\begin{description}[type=1]
\item[Source] <1-> 
\emph{Tripadvisor} website (\url{www.tripadvisor.com})
\item[Extractor] <2-> 
\emph{Visual Web Ripper} %(\url{www.visualwebripper.com})  
\item[Data] <3-> 
\begin{itemize}
 \item<4-> $8561$ records about ratings on \emph{Singapore} hotel \emph{features} and overall \emph{ratings};
 \item<5-> \emph{Demographic}, \emph{Region}, \emph{Travel Types}, etc
\end{itemize} 
\end{description}
% \medskip
\end{slide}
%%
%%==========================================================================================

%%==========================================================================================
%%  
\begin{slide}[toc=,bm=]{Data Collection}
\begin{itemize}
 \item<1-> Hotel Rating Data Collections:
\end{itemize}
\onslide*{2-}
{ \footnotesize \centering
\begin{tabular}{cccr}
\toprule
\textbf{Travel Type} & \textbf{Region} & \multicolumn{1}{c}{\textbf{Size}}\\
\midrule
\multirow{4}{*}{Business}  & Asia & $1210$ instances\\
  & Europe & $581$ instances\\
  & North America & $407$ instances \\
  & Oceania & $381$ instances\\
\midrule
      \multirow{4}{*}{Couple}  & Asia & $1169$ instances\\
 & Europe & $1389$ instances\\
 &North America & $320$ instances\\
  & Oceania & $1188$ instances\\
\midrule
        \multirow{4}{*}{Family}  & Asia & $951$ instances\\
 & Europe & $309$ instances\\
& North America & $131$ instances\\
  & Oceania & $525$ instances\\
\midrule
\multicolumn{2}{r}{\textbf{Total:}} & \textbf{8561} instances \\
\bottomrule
\end{tabular}
}
\end{slide}
%%
%%==========================================================================================

%%==========================================================================================
%%  
\begin{slide}[toc=,bm=]{Model Evaluation}
\begin{itemize}
 \item<1-> \emph{Choquet Integral} is evaluated against typical aggregation methods such as AM, WAM and OWA;
 \item<2-> Evaluation using the \emph{mean absolute  error} (MAE) based on $10$-fold cross validation.
\end{itemize}
\onslide*{2-}
{ \footnotesize \centering
\begin{tabular}{c c c c c}
\toprule
&   &\multicolumn{3}{c}{\textbf{Data Sets}}\\
\multicolumn{2}{c}{\textbf{Algorithms}}   & Business  & Couple & Family     \\
\midrule
\multicolumn{2}{c}{\textbf{AM}}    &$0.0757$ & $0.0689$& $0.0706$ \\
\multicolumn{2}{c}{\textbf{OWA}}     &$0.0718$ & $0.0666$& $0.0682$ \\
\multicolumn{2}{c}{\textbf{WAM}}    &$0.0701$ & $0.0633$& $0.0665$ \\
\midrule
\multirow{6}{*}{\textbf{CI}}
& $kadd = 1$ &     $0.0701$ & $0.0633$& $0.0665$ \\
& $kadd = 2$ &    $0.0661$ & $0.0614$& $0.0646$ \\
& $kadd = 3$ &     $0.0618$ & $0.0556$& $0.0611$ \\
& $kadd = 4$ &     $0.0619$ & $0.0561$& $0.0620$ \\
& $kadd = 5$ &     $0.0579$ & $0.0541$& $0.0598$ \\
& $kadd = 6$ &     \underline{0.0576} & \underline{0.0540}& \underline{0.0591} \\
\bottomrule
\end{tabular}
}
\end{slide}
%%
%%==========================================================================================

\section{Conclusion}


\begin{slide}[toc=,bm=]{Related Resources}
\begin{thebibliography}{1}
\bibitem{LLVR12} Gang Li, Rob Law, Huy Quan Vu, Jia Rong. 
Discovering the Hotel Selection Preferences of Hong Kong Inbound Travelers Using the Choquet Integral, 
\emph{Tourism Management}, Accepted on 21/10/2012
\footnote{An extensive invesigation on \emph{HK} hotels preferences.}

\bibitem{VBL12} Huy Quan Vu, Gleb Beliakov, Gang Li. 
A Choquet Ingtegral Toolbox and its Application in Customer's Preference Analysis, 
in \emph{Data Mining Applications with R}, Book Chapter, \emph{Elsevier} 2013
\footnote{RFMTool usage and a case study on \emph{Singapore} hotels preferences.}
\footnote{RFMTool: A R toolbox package publically available at TULIP Portal: \url{http://www.tulip.org.au/resources/rfmtool}}
\end{thebibliography}
\end{slide}


\begin{slide}[toc=,bm=]{Conclusion}
\begin{itemize}
\item This work has extended practical capability of feature-based sentiment mining; \pause

\item introduced a new aggregation techniques for effective MCMD modeling; \pause
 
\item constructed a hotel preference profile for travels from different areas and under different travel modes.
This will be also helpful for hotel managers in strategic planning and decision making.
\end{itemize}
\pause
\begin{center}
\textcolor{red}{\scalebox{1.5}{Questions?}}
\end{center}
\end{slide}


%%==========================================================================================
% TODO: Contact Page
\begin{wideslide}[toc=,bm=]{Contact Information}
\centering
\vspace{\stretch{1}}
\twocolumn[
lcolwidth=0.35\linewidth,
rcolwidth=0.65\linewidth
]
{
% \centerline{\includegraphics[scale=.2]{tulip-logo.eps}}
}
{
\vspace{\stretch{1}}
Associate Professor \emph{Gang Li}\\
School of Information Technology\\
Deakin University, Geelong, Australia
\begin{description}
 \item[Email] \href{mailto:gangli@acm.org}
 {\textsc{\footnotesize{gangli@acm.org}}}
 
 \item[Lab] \href{http://www.tulip.org.au}
 {\textsc{\footnotesize{Team for Universal Learning and Intelligent Processing}}} 
\end{description}
} 
\vspace{\stretch{1}}
\end{wideslide}



\end{document}

\endinput
