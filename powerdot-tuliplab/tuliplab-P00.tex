% 
% ---------------------------------------------------------------
% Copyright (C) 2012-2018 Gang Li
% ---------------------------------------------------------------
%
% This work is the default powerdot-tuliplab style test file and may be
% distributed and/or modified under the conditions of the LaTeX Project Public
% License, either version 1.3 of this license or (at your option) any later
% version. The latest version of this license is in
% http://www.latex-project.org/lppl.txt and version 1.3 or later is part of all
% distributions of LaTeX version 2003/12/01 or later.
%
% This work has the LPPL maintenance status "maintained".
%
% This Current Maintainer of this work is Gang Li.
%
%

\documentclass[
 size=12pt,
 paper=smartboard, %a4paper, smartboard, screen
 mode=present, %present, handout, print
 display=slides, % slidesnotes, notes, slides
% nohandoutpagebreaks,
% pauseslide,
style=tuliplab,
% nopagebreaks,clock
% hlentries=true,
% hlsections = true,
pauseslide,
fleqn,leqno]{powerdot}

\hypersetup{pdfpagemode=FullScreen}
% \usepackage[toc,highlight,blackslide,slidesonly,sounds,HA]{HA-prosper}

\usepackage{amssymb}
\usepackage{amsmath} 
\usepackage{rotating}
\usepackage{graphicx}
\usepackage{boxedminipage}
\usepackage{media9}
\usepackage{rotate}
\usepackage{calc}
\usepackage[absolute]{textpos}
\usepackage{psfrag,overpic}
\usepackage{fouriernc}
\usepackage{pstricks,pst-node,pst-text,pst-3d,pst-grad}
\usepackage{moreverb,epsfig,color,subfigure}
\usepackage{color}
\usepackage{pstricks}
\usepackage{pstricks-add}
\usepackage{pst-text}
\usepackage{pst-node, pst-tree}
\usepackage{booktabs}
\usepackage{etex}
\usepackage{breqn}
\usepackage{multirow}
\usepackage{gitinfo2}


\usepackage{listings}
\lstset{frameround=fttt, 
frame=trBL, 
stringstyle=\ttfamily,
backgroundcolor=\color{yellow!20},
basicstyle=\footnotesize\ttfamily}
\lstnewenvironment{code}{
\lstset{frame=single,escapeinside=`',
backgroundcolor=\color{yellow!20},
basicstyle=\footnotesize\ttfamily}
}{}


\usepackage{fouriernc}
\usepackage{hyperref}

%%%%%%%%%%%%%%%%%%%%%%%%%%%%%%%%%%%%%%%%%%%%%%%%%%%%%%%%%%%%%%%%%%%%%%%%
% title
% TODO: Customize to your Own Title, Name, Address
%
\title{Presentation Title}
\author{
Gang Li
\\
Deakin University 
% \href{mailto:gangli@acm.org}{gangli@acm.org}
% \and % more authors
}
\date{\gitCommitterDate}


% Customize the setting of slides
\pdsetup{
% theslide=\arabic{slide}~/~\pageref*{lastslide},
% theslide=\arabic{slide},
rf=\href{http://www.tulip.org.au}{
Last Changed by: \textsc{\gitCommitterName}\ \gitVtagn-\gitAbbrevHash\ (\gitAuthorDate)
},
cf=\hyperlink{blankslide}{Powerdot Example},
trans=Fade,
list={labelsep=1em,leftmargin=*,itemsep=0pt,topsep=5pt,parsep=0pt},
% counters={theorem,lemma},
% randomdots,dmaxdots=80
}


\begin{document}

\maketitle 

\begin{slide}[toc=,bm=]{Overview}
\tableofcontents[content=sections]
\end{slide}
\section{First section}
\begin{slide}[toc=,bm=]{Overview of the first section}
\tableofcontents[content=currentsection,type=1]
\end{slide}
\begin{slide}{Some slide}
\end{slide}

\begin{slide}{Content}
  \begin{itemize}
    \item Introduction\pause
    \item Background
    \item Method \pause
      \begin{itemize}
      \item Algorithm
      \item Proof
      \end{itemize}
    \item Experiment
    \item Conclusions
  \end{itemize}
\end{slide}


\section{Itemize Pause}


\begin{slide}{Pause longer}
\begin{itemize}
\item A \pause
\item B \pause[2]
\item C
\end{itemize}
\end{slide}


\begin{slide}{Multiple pauses}
power\pause dot \pause
\begin{itemize}
\item Let me pause\ldots \pause
\item \ldots while I talk \pause and chew bubble gum. \pause
\item Perhaps you’ll be persuaded.
\item Perhaps not.
\end{itemize}
\end{slide}


\begin{slide}{Type 1 itemize}
\begin{itemize}[type=1]
\item A \pause
\item B \pause
\item C
\end{itemize}
\end{slide}

\begin{slide}{Type 1 enumerate}
\begin{enumerate}[type=1]%[label=\romani*)]
\item A \pause
\item B \pause
\item C
\end{enumerate}
\end{slide}


\begin{slide}{Nested lists}
\begin{itemize}
\item A\pause
\begin{itemize}[type=1]
\item B\pause
\end{itemize}
\item C
\end{itemize}
\end{slide}

\section{Overlay}

\begin{slide}{Active itemize}
\begin{itemize}[type=1]
\item<1> A
\item<2> B
\item<3> C
\end{itemize}
\end{slide}

\begin{slide}{Simple onslide}
\onslide{1,2}{power}\onslide{2}{dot}
\end{slide}

\begin{slide}{Simple onslide+}
\texttt{onslide }: \onslide{1}{power}\onslide{2}{dot}\\
\texttt{onslide+}: \onslide+{1}{power}\onslide+{2}{dot}
\end{slide}

\begin{slide}{Simple onslide*}
\texttt{onslide }: \onslide{1}{power}\onslide{2}{dot}\\
\texttt{onslide+}: \onslide+{1}{power}\onslide+{2}{dot}\\
\texttt{onslide*}: \onslide*{1}{power}\onslide*{2}{dot}
\end{slide}

\begin{slide}{Lists}
\onslide{10}{on overlay 10 only}\par
\onslide{-5}{on every overlay before and including overlay 5}\par
\onslide{5-}{on every overlay after and including overlay 5}\par
\onslide{2-5}{on overlays 2 through 5, inclusive}\par
\onslide{-3,5-7,9-}{on every overlay except overlays 4 and 8}
\end{slide}

\begin{slide}{Relative overlays}
\begin{itemize}
\item A \pause
\item B \onslide{+1}{(visible 1 overlay after B)}\pause
\item C \onslide{+2-}{(appears 2 overlays after C, visible until the end)}
\pause
\item D \onslide{+1-6}{(appears 1 overlay after D, visible until overlay 6)}
\pause
\item E \pause
\item F \pause
\item G \onslide{+1-+3}{(appears 1 overlay after G for 3 overlays)}\pause
\item H \pause
\item I \pause
\item J \pause
\item K
\end{itemize}
\end{slide}


\begin{slide}{Slide 2}
  \begin{itemize}
    \item<1-> Here
    \begin{itemize}
      \item<2-> we
      \begin{itemize}
        \item<3-> demonstrate
        \begin{itemize}
          \item<4-> the itemize environment
        \end{itemize}
      \end{itemize}
    \end{itemize}
  \end{itemize}
\end{slide}

\begin{slide}{Slide 3}
  \begin{enumerate}[type=1]
    \item<1> Here
    \begin{enumerate}
      \item<2> we
      \begin{enumerate}
        \item<3> demonstrate
        \begin{enumerate}
          \item<4> the enumerate environment
        \end{enumerate}
      \end{enumerate}
    \end{enumerate}
  \end{enumerate}
\end{slide}


\section{Codings}

\begin{slide}[method=direct]{Slide 2}
Steps 1 and 2:
\begin{code}
compute a;`\pause'
compute b;
\end{code}
\end{slide}

\begin{slide}[method=file]{Slide 3}
Steps 1 and 2:
\begin{code}
compute a;`\pause'
compute b;
\end{code}
\end{slide}


%% Coding Example
\begin{slide}[method=direct]{Hello World in C}
  \begin{lstlisting}[language=c,gobble=4]
    #include <stdio.h>

    int main(int c, char **v) {
      fprintf(stdout, "Hello, world!\n");
      return 0;
    }
  \end{lstlisting}
\end{slide}
\begin{note}{Personal Note}
 This is a test. Let us try
\begin{description}
 \item[ab] ab 
 \item[bb] ab 
 \item[bb] ab 
\end{description}
\end{note}

\section{Citations}

\begin{slide}{Cite}
\cite{someone}
\end{slide}

\begin{slide}{References}
\begin{thebibliography}{1}
\bibitem{someone} Article of someone.
\end{thebibliography}
\end{slide}

% \begin{slide}{References (1)}
% \bibliographystyle{plain}
% \nobibliography{YourBib}
% \bibentry{someone1}
% \bibentry{someone2}
% \end{slide}
% \begin{slide}{References (2)}
% \bibentry{someone3}
% \end{slide}

% \begin{slide}{Slide}
% \cite{someone}
% \end{slide}
% 
% \begin{slide}{References}
% \bibliographystyle{plain}
% % \bibliography{YourBib}
% \end{slide}


\section{Two Columns}

\begin{slide}{Two columns}
Here are two columns.

\twocolumn[
lfrprop={linestyle=dotted,linewidth=3pt},
lfrheight=4cm,rfrheight=5cm,lineheight=3cm,topsep=0.3cm
]{left}{right}

% \hline
Another two columns:

\twocolumn[
\savevalue{lfrheight}=3cm,
\savevalue{lfrprop}={
linestyle=dotted,framearc=.2,linewidth=3pt},
rfrheight=\usevalue{lfrheight},
rfrprop=\usevalue{lfrprop}
]{left}{right}


Those were two columns.
\end{slide}

\begin{slide}[method=direct]{Two Columns}
  \begin{lstlisting}[language=c,gobble=4]
    #include <stdio.h>

    int main(int c, char **v) {
      fprintf(stdout, "Hello, world!\n");
      return 0;
    }
  \end{lstlisting}
Another two columns:

\twocolumn[
\savevalue{lfrheight}=3cm,
\savevalue{lfrprop}={
linestyle=dotted,framearc=.2,linewidth=3pt},
rfrheight=\usevalue{lfrheight},
rfrprop=\usevalue{lfrprop}
]{
\begin{itemize}
\item A \pause
\item B \pause[2]
\item C
\end{itemize}
}
{
     \centerline{\includegraphics[scale=.3]{logos/tulip-logo.eps}}
}

\end{slide}



\section{Misc}

\begin{emptyslide}{}
\centering
\vspace{\stretch{1}}
\includegraphics[height=0.8\slideheight]{logos/tulip-logo.eps}
\vspace{\stretch{1}}
\end{emptyslide}


\begin{slide}[toc=,bm={LaTeX, i*i=-1}]{\color{red}\LaTeX, $i^2=-1$}
My slide contents.
\end{slide}

%%==========================================================================================
% TODO: Contact Page
\begin{wideslide}[toc=,bm=]{Contact Information}
\centering
\vspace{\stretch{1}}
\twocolumn[
lcolwidth=0.35\linewidth,
rcolwidth=0.65\linewidth
]
{
% \centerline{\includegraphics[scale=.2]{tulip-logo.eps}}
}
{
\vspace{\stretch{1}}
Associate Professor \emph{Gang Li}\\
School of Information Technology\\
Deakin University, Geelong, Australia
\begin{description}
 \item[Email] \href{mailto:gangli@acm.org}
 {\textsc{\footnotesize{gangli@acm.org}}}
 
 \item[Lab] \href{http://www.tulip.org.au}
 {\textsc{\footnotesize{Team for Universal Learning and Intelligent Processing}}} 
\end{description}
} 
\vspace{\stretch{1}}
\end{wideslide}



\end{document}

\endinput
