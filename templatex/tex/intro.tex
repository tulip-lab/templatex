%=================================================================
\section{Introduction}\label{sec-intro}


%\todo{Narrow down to a topic; Dig a hole; Fill the hole}
%\todo{Formula for Introduction}

\lipsum[3]


%\gangli{``narrow in on topic'' reminds you 
%that readers and reviewers only know that this is a AI or HTM research paper (and maybe have read the title/abstract). 
%You need to help them figure out what topic and area of research paper this is. 
%You _don't_ need to wax poetic about the topic's importance.}

%\gangli{`dig a hole'' reminds you that 
%you need to convince the reader that there's a problem with the state of the world. 
%Prior work may exist but it's either missing something important or there's a missing opportunity. 
%The reader should be drooling for a bright future just out of reach.}

%\gangli{``fill the hole'' reminds you to show the reader 
%how and why the paper they're reading will fix these problems and deliver us into a better place. 
%You don't need a whirlwind summary of the technical details, 
%but you need readers convinced (and in a good mood) to keep reading.}

%\gangli{A good paper introduction is fairly formulaic. 
%If you follow a simple set of rules, 
%you can write a very good introduction. 
%The following outline can be varied. 
%For example, 
%you can use two paragraphs instead of one, 
%or you can place more emphasis on one aspect of the intro than another. 
%But in all cases, 
%all of the points below need to be covered in an introduction, 
%and in most papers, 
%you don't need to cover anything more in an introduction.}
%
%
%
%%\todo{The importance of the area}
%%\blindtext
%\todo{Motivation}
%At a high level, 
%what is the problem area you are working in and why is it important? 
%It is important to set the larger context here. 
%Why is the problem of interest and importance to the larger community?
%
%
%%\todo{The problems faced by most current methods}
%%\blindtext
%\todo{What is the specific problem considered in this paper?}
%This paragraph narrows down the topic area of the paper. 
%In the first paragraph you have established general context and importance. 
%Here you establish specific context and background.
%
%%\todo{What can be addressed by existing methods; Why those problems are challenges to existing methods?}
%%\blindtext
%\todo{Contribution}
%``In this paper, we show that ...''
%This is the key paragraph in the intro - you summarize, 
%in one paragraph, 
%what are the main contributions of your paper given the context 
%you have established in paragraphs 1 and 2. 
%What is the general approach taken? 
%Why are the specific results significant? 
%This paragraph must be really good. 
%
%You should think about how to structure these one or 
%two paragraph summaries of what your paper is all about. 
%If there are two or three main results, 
%then you might consider itemizing them with bullets or in test. 
%\begin{itemize}
%	\item e.g., First ...
%	\item e.g., Second ...
%	\item e.g., Third ...
%\end{itemize}
%If the results fall broadly into two categories, 
%you can bring out that distinction here. 
%For example, 
%``Our results are both theoretical and applied in nature. 
%(two sentences follow, one each on theory and application)''
%
%%\todo{What provides the motivation of this work? What are the research issues? What is the rationale of this work? }
%%\blindtext
%\todo{At a high level what are the differences in what you are doing, and what others have done? }
%Keep this at a high level, 
%you can refer to a future section where specific details and differences will be given. 
%But it is important for the reader to know at a high level, 
%what is new about this work compared to other work in the area.
%
%%\todo{What we have done and what are the contributions.}
%%\blindtext
%\todo{A roadmap for the rest of the paper}
%"The remainder of this paper is structured as follows..." 
%Give the reader a roadmap for the rest of the paper. 
%Avoid redundant phrasing, 
%"In Section 2, In section 3, ... In Section 4, ... " etc.

%\gangli{A few general tips:
%Don't spend a lot of time into the introduction 
%telling the reader about what you don't do in the paper. 
%Be clear about what you do do.
%Does each paragraph have a theme sentence that sets the stage for the entire paragraph? Are the sentences and topics in the paragraph all related to each other?}

%\gangli{Does each paragraph have a theme 
%sentence that sets the stage for the entire paragraph? 
%Are the sentences and topics in the paragraph all related to each other?}

\gangli{Do all of your tenses match up in a paragraph?}

%Test citation~\cite{BL12J01}. 
%\begin{JournalOnly}
%and~\citep{BJL11J01} or~\citet{BJL11J01}.
%\end{JournalOnly}
%
%This is for~\cref{tbl:overall-experiments}, 
%\todo[fancyline]{Testing.}
%and this is for~\cref{sec-conclusions}.
%\todo[noline]{A note with no line back to the text.}%
%\gangli{This is comment from Gang.}
%\qwu{Response from QW}
%
%Number:
%\num{123}.
%\numlist{10;30;50;70},
%\numrange{10}{30},
%\SIlist{10;30;45}{\metre},
%and
%\SI{10}{\percent}
%
%\missingfigure[figcolor=white]{Testing figcolor}
%
%
%\begin{ConferenceOnly}
%We have \SI{10}{\hertz},
%\si{\kilogram\metre\per\second},
%the range: \SIrange{10}{100}{\hertz}.
%$\nicefrac[]{1}{2}$.
%
%\missingfigure{Make a sketch of the structure of a trebuchet.}
%
%\end{ConferenceOnly}
%
%
%For~\cref{eq:test},
%as shown below:
%
%\begin{equation}\label{eq:test}
%a = b \times \sqrt{ab}
%\end{equation}
%
%\blindmathpaper
%
%\section{Preliminaries} \label{sec-preliminaries}
%
%\blindtext
%
%\gliMarker  %TODO: GLi Here
%
%
%\section{Method} \label{sec-method}
%
%%\blindtext
%\blindlist{itemize}[3]
%\blinditemize
%%\blindenumerate
%
%\blindmathtrue
%%\blindmathfalse
%\blinddescription

%\qwuMarker %TODO: QWu Here
%
%\section{Experiment and Analysis} \label{sec-experiment}
%
%
%\begin{table}  \centering
%  \caption{Precision Comparison on Event Detection Methods}
%  \label{tbl:overall-experiments}
%  \begin{tabular}{cccc}
%\toprule
%    % after \\: \hline or \cline{col1-col2} \cline{col3-col4} ...
%    & OR Event Detection & AC Event Detection & TC Event Detection \\
%\midrule
%    precision & 0.83 & 0.69 & 0.46 \\
%    recall & 0.68 & 0.48 & 0.36 \\
%    F-score & 0.747 & 0.57 & 0.4 \\
%\bottomrule
%\end{tabular}
%\end{table}

