\documentclass{amsart}
\synctex=1

%=================================================================
% gitlatexdiff
%
%  https://gitlab.com/git-latexdiff/git-latexdiff
%=================================================================
%  git latexdiff HEAD  HEAD~5 --main templatex.tex
%  git latexdiff HEAD~1  --main templatex.tex
%  View pdf to see difference
%
%=================================================================
%
% Todo Notes for marginal comments
%
\usepackage{marginnote}
% 
\newcount\DraftStatus  % 0 suppresses notes to selves in text
\DraftStatus=1   % TODO: set to 0 for final version
\ifnum\DraftStatus=1
	\usepackage[draft,colorinlistoftodos,color=orange!30]{todonotes}
\else
	\usepackage[disable,colorinlistoftodos,color=blue!30]{todonotes}
\fi 
%\usepackage[disable]{todonotes} % notes not showed
%\usepackage[draft]{todonotes}   % notes showed
%
\makeatletter
 \providecommand\@dotsep{5}
 \def\listtodoname{List of Todos}
 \def\listoftodos{\@starttoc{tdo}\listtodoname}
 \makeatother
%
%=================================================================
%
\usepackage{color}
\newcommand{\draftnote}[3]{ 
	\todo[author=#2,color=#1!30,size=\footnotesize]{\textsf{#3}}	}
% TODO: add yourself here:
%
\newcommand{\gangli}[1]{\draftnote{blue}{GLi:}{#1}}
\newcommand{\qwu}[1]{\draftnote{red}{QWu:}{#1}}
\newcommand{\gliMarker}
	{\todo[author=GLi,size=\tiny,inline,color=blue!40]
	{Gang Li has worked up to here.}}
\newcommand{\qwuMarker}
	{\todo[author=QWu,size=\tiny,inline,color=red!40]
	{Qiong Wu has worked up to here.}}
%=================================================================
% for including/hiding
% TODO: revise here for versions
\usepackage{comment}
\includecomment{JournalA}  
\excludecomment{JournalB}
%
%=================================================================


%=================================================================
%
% general packages
%  https://en.wikibooks.org/wiki/Category:Book:LaTeX
%  https://en.wikibooks.org/wiki/LaTeX/Package_Reference
%
%=================================================================
\usepackage{graphicx}
\usepackage{algorithm}
\usepackage{algorithmic}
\usepackage{breqn}
\usepackage{subfigure}
\usepackage{multirow}
\usepackage{psfrag}
\usepackage{url}
\usepackage[colorlinks]{hyperref}
\usepackage{natbib}
\usepackage{booktabs}
\usepackage{rotating}
\usepackage{colortbl}
\usepackage{spelling}
%\usepackage{paralist}
\usepackage{geometry}
%\usepackage{epstopdf}
\usepackage{nag}
\usepackage{microtype}
\usepackage{siunitx}
\usepackage{cleveref}
%\usepackage{booktabs}
\usepackage{nicefrac}
% for random text
\usepackage{lipsum}
\usepackage[english]{babel}
\usepackage[pangram]{blindtext}
%
%=================================================================


%=================================================================
%
% Version control information
%
%=================================================================
\usepackage{gitinfo2}
%=================================================================
\usepackage{fancyhdr}
\pagestyle{fancy}
\fancyhead{} % clear all header fields
\fancyhead[RO,LE]{\textsl{\rightmark}}
\fancyhead[LO,RE]{\ensuremath{\Rightarrow}
		\textbf{\textbf{[CONFIDENTIAL]}}\ensuremath{\Leftarrow}}
\fancyhead[CO,CE]{}
%=================================================================
\fancyfoot{} % clear all footer fields
\fancyfoot[CE,CO]{\textbf{\thepage}} 
\fancyfoot[LO,LE]{\includegraphics[height=.9\headheight]{figures/tulip-icon.png}
		\gitVtagn-\gitBranch\ (\gitCommitterDate)}
\fancyfoot[RO,RE]{Committed by: \textsl{\gitCommitterName}}

\setlength{\headheight}{12pt}
\renewcommand{\headrulewidth}{0.4pt}
\renewcommand{\footrulewidth}{0.4pt}
%=================================================================

%=================================================================
% for math notations
% ----------------------------------------------------------------
\usepackage{mathtools}
\usepackage{amsthm}
%
% THEOREMS -------------------------------------------------------
%
\newtheorem{thm}{Theorem}[section]
\newtheorem{cor}[thm]{Corollary}
\newtheorem{lem}[thm]{Lemma}
\newtheorem{prop}[thm]{Proposition}
\theoremstyle{definition}
\newtheorem{defn}[thm]{Definition}
\theoremstyle{remark}
\newtheorem{rem}[thm]{Remark}
\numberwithin{equation}{section}
% MATH -----------------------------------------------------------
\newcommand{\norm}[1]{\left\Vert#1\right\Vert}
\newcommand{\abs}[1]{\left\vert#1\right\vert}
\newcommand{\set}[1]{\left\{#1\right\}}
\newcommand{\Real}{\mathbb R}
\newcommand{\eps}{\varepsilon}
\newcommand{\To}{\longrightarrow}
\newcommand{\BX}{\mathbf{B}(X)}
% ----------------------------------------------------------------
\newcommand{\I}{{\cal I}}
\newcommand{\Id}{{\cal I} }
\newcommand{\Dc}{{\cal D}}
\newcommand{\J}{{\cal J}}
\newcommand{\Dn}{{\cal D}_n}
\newcommand{\Dd}{{\cal D}_n }
\renewcommand{\P}{{\cal P}}
\newcommand{\Nu}{{\cal N} }
\newcommand{\B}{{\cal B}}
\newcommand{\Bf}{{\bf B}}
\newcommand{\Y}{{\bf Y}}
\newcommand{\A}{{\cal A}}
% ----------------------------------------------------------------
\newcommand{\V}{{\cal V}}
\newcommand{\M}{{\cal M}}
\newcommand{\F}{{\cal F}}
\newcommand{\Fd}{{\cal F}}
\newcommand{\BF}{{\cal BF}_n}
\newcommand{\BFd}{{\cal BF}_n}
\newcommand{\TF}{{\cal TF}_n}
\newcommand{\TFd}{{\cal TF}_n}
\newcommand{\G}{{\cal G}}
\newcommand{\X}{{\cal X}}
\newcommand{\E}{{\cal E}}
\newcommand{\K}{{\cal K}}
\newcommand{\T}{{\cal T}_n}
\renewcommand{\H}{{\cal H}}
% ----------------------------------------------------------------
\newtheorem{Remark}{Remark}
\newtheorem{proposition}{Proposition}
\newtheorem{theorem}{Theorem}
\newtheorem{lemma}{Lemma}
\newtheorem{corollary}{Corollary}
\newtheorem{example}{Example}
\newtheorem{definition}{Definition}
\newtheorem{Algorithms}{Algorithm}
% ----------------------------------------------------------------
\newcommand{\bu}{{\mathbf 1} }
\newcommand{\bo}{{\mathbf 0} }
\newcommand{\N}{\mbox{{\sl l}}\!\mbox{{\sl N}}}
% ----------------------------------------------------------------
\def\uint{[0,1]}
\def\proof{{\scshape Proof}. \ignorespaces}
\def\endproof{{\hfill \vbox{\hrule\hbox{%
   \vrule height1.3ex\hskip1.0ex\vrule}\hrule
  }}\par}
%
%=================================================================
\hypersetup
{
    pdfauthor={\gitAuthorName},
    pdfsubject={TULIP Lab},
    pdftitle={},
    pdfkeywords={TULIP Lab, Data Science},
%	bookmarks=true,  
}


%=================================================================
%
\begin{document}
%
%=================================================================
% Preamble which will need to be changed for submission
%
\title[A Short Running Title]{Title of This Paper}%

\author{Author 1}
\address[A.~1]{School of Information Technology \\
Deakin University, 221 Burwood Highway \\
Vic 3125, Australia}%
\email[A.~1]{xxx@deakin.edu.au}

\author{Gang Li}
\address[A.~2]{School of Information Technology \\
Deakin University, 221 Burwood Highway \\
VIC 3125, Australia}%
\email[A.~2]{gang.li@deakin.edu.au}

\author{Author 3}
\address[A.~3]{School of Information Technology \\
Deakin University, 221 Burwood Highway \\
Vic 3125, Australia}%
\email[A.~3]{xxx@deakin.edu.au}

%\thanks{Thanks to \ldots}%
\subjclass{Artificial Intelligence}%
\keywords{Machine Learning, Data Mining, ...}%
\date{\gitAuthorDate}%

\begin{abstract}
The abstract will be put here, ....
\end{abstract}

\maketitle
\tableofcontents

\newpage
%=================================================================


\section{Introduction}\label{sec-intro}

Test citation~\cite{BJL11J01}, and~\citep{BJL11J01} or~\citet{BJL11J01}.
\gangli{This is comment from Gang.}
\qwu{Response from QW}
This is for~\cref{tbl:overall-experiments}, 
\todo[fancyline]{Testing.}
and this is for~\cref{sec-conclusions}.
\todo[noline]{A note with no line back to the text.}%

\begin{JournalA}
Number:
\num{123}.
\numlist{10;30;50;70},
\numrange{10}{30},
\SIlist{10;30;45}{\metre},
and
\SI{10}{\percent}

\missingfigure[figcolor=white]{Testing figcolor}

\end{JournalA}

\begin{JournalB}
We have \SI{10}{\hertz},
\si{\kilogram\metre\per\second},
the range: \SIrange{10}{100}{\hertz}.
$\nicefrac[]{1}{2}$.

\missingfigure{Make a sketch of the structure of a trebuchet.}

\end{JournalB}


For~\cref{eq:test},
as shown below:

\begin{equation}\label{eq:test}
a = b \times \sqrt{ab}
\end{equation}

\blindmathpaper

\section{Preliminaries} \label{sec-preliminaries}

\blindtext

\gliMarker


\section{Method} \label{sec-method}


% generates a paragraph of dummy lorem ipsum text
 
%\blindtext 
 
 
% generates multiple paragraphs of dummy lorem ipsum text
 
%\Blindtext 
 
 
% generates whole document with dummy lorem ipsum text
 
%\Blinddocument 
\blindtext

\blindlist{itemize}[3]
\blinditemize
\blindenumerate

\blindmathtrue
%\blindmathfalse
\blinddescription

\qwuMarker

\section{Experiment and Analysis} \label{sec-experiment}


\begin{table}  \centering
  \caption{Precision Comparison on Event Detection Methods}
  \label{tbl:overall-experiments}
  \begin{tabular}{cccc}
\toprule
    % after \\: \hline or \cline{col1-col2} \cline{col3-col4} ...
    & OR Event Detection & AC Event Detection & TC Event Detection \\
\midrule
    precision & 0.83 & 0.69 & 0.46 \\
    recall & 0.68 & 0.48 & 0.36 \\
    F-score & 0.747 & 0.57 & 0.4 \\
\bottomrule
\end{tabular}
\end{table}


\section{Conclusions} \label{sec-conclusions}

\blindtext

\section*{Acknowledgment}

\lipsum[1]


The authors would like to thank \ldots

% ----------------------------------------------------------------
\newpage
%\bibliographystyle{plain}
\bibliographystyle{newapa}
\bibliography{tulip}
%=================================================================

\listoftodos


\end{document}

